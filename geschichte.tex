%!TEX root = thesis.tex

\chapter{Geschichte}
\label{chapter-geschichte}

Die heutige Straße An der Untertrave wird erst seit dem späten 19. Jahrhundert als durchgehender Straßenzug aufgefasst. Zuvor führten nur einzelne Abschnitte eigene Bezeichnungen:

\begin{compactitem}
\item Der Bereich an der Einmündung der Fischstraße hieß Eisenmarkt, da in diesem Teil des Hafens das aus Schweden eingeführte Stangeneisen angelandet und gelagert wurde. Die älteste überlieferte Form des Namens ist aus dem Jahr 1700 bekannt und lautete Osemundsmarkt
\item Das Straßenstück zwischen Alsheide und Alfstraße hieß seit dem 16. Jahrhundert Petri-Sanddamm, da hier ein zur Petrikirche gehöriges Sandhaus stand
\item Bei der Kleinen Altefähre lautete der Straßenname 1319 Apud arborem inferiorem (lat. Beim unteren Baum) und 1459 To dem Torne, da hier eine schwimmende Sperre aus Baumstämmen die Hafeneinfahrt abriegelte
\item Der Abschnitt zwischen Alfstraße und Mengstraße war 1841 als Weinstaat bekannt, weil hier die mit importierten Weinen beladenen Schiffe festmachten
\item Die Sektion zwischen Beckergrube und Fischergrube wurde seit spätestens 1483 Heringsmarkt genannt
\item Nachdem die Anleger zwischen Große Altefähre und Burgtor 1834 zu einem allein Dampfschiffen vorbehaltenen Hafen ausgebaut worden waren, erhielt der hier verlaufende Straßenzug den Namen Am Dampfschiffshafen
\end{compactitem}

Im Jahre 1884 wurde der gesamte Straßenzug zusammengefasst und erhielt seinen bis heute gültigen Namen.

Die nunmehr sehr breite Untertrave war ursprünglich eine schmale, beengte Straße, die zwischen den langen Reihe der Speicher Häuser auf der östlichen und der mittelalterlichen Stadtmauer auf der westlichen Seite verlief. Der Hafen lag außerhalb der Mauer und war von der Straße her durch Tordurchlässe zugänglich. Reste der Stadtmauer waren entlang der gesamten Untertrave bis in die Mitte des 19. Jahrhunderts vorhanden, dann wurden sie im Zuge des Hafenausbaus abgebrochen. Auf der Ostseite der Untertrave wurden seit 1853 zeitgemäße Lagerhallen errichtet sowie Bahngleise für den Güterverkehr verlegt. Die Hafenbahn, die noch zu Beginn der 1950er Jahre die gesamte Untertrave bis hinab zur Holstenbrücke säumte, führte zuletzt nur noch bis zur Drehbrücke gegenüber der Engelsgrube, auf der sie die Trave überquerte.

Nach dem Zweiten Weltkrieg wurde die Straße in den fünfziger und sechziger Jahren erheblich verbreitert und mehrspurig ausgebaut, um auf ihr den rapide anwachsenden Autoverkehr um die Altstadt herumleiten zu können. Der Fußweg an der Kaikante wurde gesondert als Wenditzufer benannt. Im Zuge der Errichtung des Europäischen Hansemuseums erfolgte 2013 der Rückbau von vier auf zwei Fahrspuren in einigen Abschnitten.

Im Bereich zwischen der Drehbrücke und der Musik- und Kongreßhalle befindet sich heute der Museumshafen.

\section{Konflikt um Winterlinden}

Zum Konflikt mit der Bürgerschaft und der Verwaltung der Stadt kam es 2016, als bekannt wurde, dass 48 am Traveufer stehende Winterlinden gefällt werden sollen. Die Bäume im Abschnitt zwischen Holstenstraße und Drehbrücke sollen für die seit 2003 geplante Umgestaltung von Fahrbahn und Travepromenade der Neuanpflanzung Japanischer Schnurbäume weichen. \cite{llll}
Mitglieder eines Aktionsbündnisses („Lübecks Linden leben lassen“) sammelten bis Mitte September innerhalb von vier Wochen in einem Bürgerbegehren mehr als 10.500 Unterschriften für einen Bürgerentscheid, um das Fällen der Linden zu verhindern. \cite{lindensave}

\section{Höhenregulierung der Straße}

Die Notwendigkeit von der Drehbrücke (Engelsgrube) bis zur Kanalmündung, bedingt durch die Hochwasserschäden der vorhergehenden Jahrzehnte, neue Kaimauern zu errichten, bot die Gelegenheit gleichzeitig eine Höhenveränderung der Kaianlagen vorzunehmen. Die dahinter liegenden Straßen hatten eine entsprechende Aufhöhung zu erfahren.

Diese betrug zum Teil einen Meter und mehr. Während man bei anderen Straßenregulierungen in der Lage war, zunächst eine Mittellinie zu wählen, war das hier, da sonst aus der Straße eine Schiefe Ebene geworden wäre, nicht möglich. Die nachfolgende Bilder zeigen die auf der genannten Strecke getroffenen Veränderungen.

\begin{figure}
  \centering
  \pgfimage[width=.5\textwidth]{pflasterungsarbeiten}
  \caption[Pflasterungsarbeiten und Höhenregulierung]{Pflasterungsarbeiten und Höhenregulierung}
  \label{fig-pflasterungsarbeiten}
\end{figure}

\begin{figure}
  \centering
  \pgfimage[width=.5\textwidth]{hoehenveraenderungen}
  \caption[Höhenveränderungen]{Höhenveränderungen}
  \label{fig-hoehenveraenderungen}
\end{figure}
