%!TEX root = thesis.tex

\chapter{Bauwerke}
\label{chapter-bauwerke}

\begin{compactitem}
\item Lagerschuppen 6, 1906 errichtete einschiffige Lagerhalle in Ziegelbauweise
\item Lagerschuppen 9, 1906 errichtete zweischiffige Lagerhalle in Ziegelbauweise
\item Europäisches Hansemuseum, An der Untertrave 1
\item Petit-Haus, An der Untertrave 3
\item An der Untertrave 21, zwischen 1600 und 1649 errichtetes Renaissance-Haus mit klassizistischer Fassade von 1808
\item An der Untertrave 27, 1899 errichtete Fassade des Historismus vor einem Neubau von 1981
\item An der Untertrave 34, Die Eiche, 1873 errichteter Kornspeicher im Stil des Historismus, erbaut für Thomas Johann Heinrich Mann
\item An der Untertrave 39, zwischen 1600 und 1624 errichtetes Renaissance-Giebelhaus
\item An der Untertrave 42, zwischen 1550 und 1624 errichtetes Renaissance-Giebelhaus mit klassizistischer Fassade von 1797 und 1857
\item An der Untertrave 52-53, zwischen 1525 und 1599 errichtetes Renaissance-Doppelhaus
\item An der Untertrave 60, spätbarockes Haus des 18. Jahrhunderts
\item An der Untertrave 61, im 16. Jahrhundert erbautes Renaissance-Giebelhaus
\item An der Untertrave 62, im 16. Jahrhundert erbautes Renaissance-Giebelhaus
\item An der Untertrave 70, 1749 errichtetes Eckhaus mit verputzter Fassade
\item An der Untertrave 86, Bürgerhaus der Renaissance
\item An der Untertrave 96, 1569 errichtetes dreistöckiges Renaissance-Fachwerkhaus, bereits ursprünglich als Mietshaus geplant und \item gebaut
\item An der Untertrave 97, neogotischer Backstein-Speicher von 1871
\item An der Untertrave 98, neogotischer Backstein-Speicher von 1870
\item Das sich anschließende Eckhaus Alfstraße 38 zeigt zur Untertrave einen eindrucksvollen Stufengiebel der Renaissance.
\end{compactitem}
