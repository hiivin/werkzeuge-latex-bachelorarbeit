%!TEX root = thesis.tex

\chapter{Einleitung}
Die Untertrave, ihre Geschichte und aktuelle Konflike sind zentrale Themen dieser (prächtigen) Arbeit. Diese wunderbare Straße, die jedem Lübecker Bürger mit einem Zugang zu der schönsten Kneipenecke, der Clemensstraße, versorgt, wird betrachtet und in aktuellen Belangen dargestellt (und wäre es eine richtige Arbeit, hätte man hier definitiv noch über die aktuellen Belange diskutiert).
Wenn man sich für die Untertrave interessiert, empfiehlt sich diverse spannende Literatur über die Straßennamen der Stadt Lübecks oder auch ein gemütlicher Spaziergang.
Den Konflikt kann man auch in diversen moderierten Gesprächsrunden, die beispielsweise an der Uni zu Lübeck stattfinden, sich von beiden Parteien beleuchten lassen.

\section{Verwandte Arbeiten}

Die Untertrave und der Konflikt über Lübecks Linden ist bereits stark diskutiert. Es empfiehlt sich ein Blick in das Literaturverzeichnis. \cite{llllweb}

\section{Aufbau der Arbeit}

Neben dieser Einleitung und der Zusammenfassung am Ende gliedert sich diese Arbeit in die folgenden vier Kapitel.
\begin{description}
  \item[\ref{chapter-lage}] beschreibt die für diese Arbeit benötigten Grundlagen. In diesem Kapitel wird das behandelte Objekt in ihrer Lage und ihrem Verlauf geschildert.
  \item[\ref{chapter-geschichte}] stellt die Geschichte der Untertrave vor. Dabei werden in chronologischer Reihenfolge die Veränderungen der Untertrave dargestellt. Außerdem werden besondere oder aktuelle diskutierte Veränderungen einzeln dargestellt.
  \item[\ref{chapter-bauwerke}] beinhaltet eine Liste nennenswerter Bauwerke, die in der Straße zu finden sind.
  \item[\ref{chapter-gaengeundhoefe}] beinhaltet eine Liste der bekannten Lübecker Gänge und Höfe, die an der Untertrave zu finden sind.
\end{description}

