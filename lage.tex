%!TEX root = thesis.tex

\chapter{Lage}
\label{chapter-lage}

Die etwa 960 Meter lange Straße An der Untertrave befindet sich im Nordwesten der Altstadtinsel. Sie verläuft als Uferstraße entlang der Trave.

Die Straße beginnt am nördlichsten Punkt der Altstadt bei der Hubbrücke über den Elbe-Lübeck-Kanal, der an dieser Stelle in die Trave mündet, und der Einmündung der Kanalstraße, die den nordöstlichen Rand der Altstadtinsel erschließt. Hier besteht auch ein Fußweg mit Treppe hinauf zum Burgtor, der Marstallweg.

In einem Bogen, der aus dem Verlauf des Traveufers resultiert, führt die Straße zunächst südwestwärts, wobei aus dem Inneren der Altstadt kommend nacheinander Kleine Altefähre, Große Altefähre, Petersilienstraße, Alsheide, Engelsgrube, Fischergrube, Clemensstraße und Beckergrube einmünden. Dann beschreibt die Untertrave einen Knick, nach dem sie südwärts weiterführt und als weitere einmündende Altstadtstraßen Mengstraße, Alfstraße, Fischstraße und Braunstraße folgen. An der Kreuzung mit der Holstenstraße und der zum Holstentor führenden Holstenbrücke endet die Untertrave schließlich. Ihre Fortsetzung entlang des Traveufers ist An der Obertrave.

